\documentclass{ximera}

\title{Problem environments (\jobname)}

\begin{document}
\begin{abstract}
  Some problem environments.
\end{abstract}
\maketitle

The problem environments are \verb|problem|, \verb|exercise|, \verb|question|, and \verb|exploration|.


\begin{exercise}
  Type $2$: $\answer{2}$.
\end{exercise}

\begin{problem}
  Type $2$: $\answer{2}$.
\end{problem}

\begin{question}
  Type $2$: $\answer{2}$.
\end{question}

\begin{exploration}
  Type $2$: $\answer{2}$.
\end{exploration}

Online these act much like theorem-like environments.

However in the PDF, the documentclass option \verb|newpage| will start
a new page at the end of each of these. Moreoever, nested problem
environments will number as sub problems in the PDF.

An example of an exercise with two nested exercises is:
\begin{exercise}
  Type $1$: $\answer{1}$.
  \begin{exercise}
    Type $2$: $\answer{2}$.
    \begin{exercise}
      Type $3$: $\answer{3}$.
    \end{exercise}
  \end{exercise}
  \begin{exercise}
    Type $2$: $\answer{2}$.
  \end{exercise}
\end{exercise}


An example of an exercise with one \verb|answer| and two questions:

\begin{exercise}
  Type $1$: $\answer{1}$.
  \begin{question}
    Type $2$: $\answer{2}$.
  \end{question}
  \begin{question}
    Type $2$: $\answer{2}$.
  \end{question}
\end{exercise}

An example of an exercise with just two questions:

\begin{exercise}
  \begin{question}
    Type $2$: $\answer{2}$.
  \end{question}
  \begin{question}
    Type $2$: $\answer{2}$.
  \end{question}
\end{exercise}


\end{document}
