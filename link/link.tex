\documentclass{ximera}

\title{Links}

\author{Bart Snapp}

\begin{document}
\begin{abstract}
  Examples of links.
\end{abstract}
\maketitle

\section*{Basic Usage}\label{sec:link:basicUsage}

With \verb|\link{url}| you make a hyperlink: \link{http://ximera.osu.edu}.

With \verb|\link[text]{url}| you display 'text', and not the url: \link[ximera]{http://ximera.osu.edu}.



Notes: 
\begin{itemize}
 \item \label{link_spaces}   Note the spaces in and after a \link[link to ximera]{http://ximera.osu.edu} in a sentence.
 \item \label{link_bold}     You might want a bold \textbf{\link[link to ximera]{http://ximera.osu.edu}}, or a link with a bold 'ximera' in it: \link[link to \textbf{ximera}]{http://ximera.osu.edu}.
 \item \label{link_footnote} The PDF puts links in footnotes\footnote{This is just a normal footnote} at the bottom of the page.
\end{itemize}

Alternatives:

With \verb|\url{url}| you make a hyperlink: \url{http://ximera.osu.edu}.

With \verb|\href{url}{text}| you display 'text', and not the url: \href{http://ximera.osu.edu}{ximera}.

With \verb|\hypertarget{label}{text}| you can create a target to link to: \hypertarget{ht_bold}{this text is important enough to be linked to}.

With \verb|\hyperlink{label}{text}| you can (NOT???) link to a label: see also \hyperlink{ht_bold}{the previous sentence}.

With \verb|\hyperlink{label}{text}| you can also link to a normal \verb|label|: see the item about \hyperlink{link_bold}{bold in links}, or the section with \hyperref[sec:link:intendedOutcome]{outcomes}.

\section*{Intended Outcome of Test}\label{sec:link:intendedOutcome}

TODO


% Compiled at \today\  at \currenttime     %% needs datetime


\end{document}
