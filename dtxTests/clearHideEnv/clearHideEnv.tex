\documentclass{ximera}
\title{Clear and Hide Environments}

%\typeout{Start loading xmPreamble.tex}%


\newcommand{\ximera}{Ximera}
\usepackage{lipsum}
% Add here extra macro's that are loaded automatically by all documents of claas 'ximera' or 'xourse' in this repo

%%
%%  Example:
%%
% \newcommand{\R}{\mathbb{R}% Only add a preamble file if it is actually necessary for the demo/test.
\begin{document}
\begin{abstract}
    This demonstrates and tests the clearEnv and hideEnv commands.    
\end{abstract}
\maketitle

{{\Huge \bfseries Last Updated: \today}} \\



\section{Basic Usage}
The \verb|\clearEnv{#1}| command will attempt to undefine any provided environment, 
regardless of how it was defined (e.g. newenvironment or newEnviron commands).
The (required) argument (\#1) is the name of the intended environment you wish to undefine.

It will also attempt to undefine any default counters that may have been defined in that environment
(e.g. problem counters). It should also allow compiles where those environments are still called later in 
the document to compile (although the environment itself will do nothing, it won't throw an error and stop the compile).\\

The \verb|\hideEnv[#1]{#2}| macro allows you to hide an environment by name, ensuring it - and all its contents - 
are no longer visible on the pdf output. It also allows you an optional input to replace the block of content with 
a new (uniform) content if desired.
The first argument (\#1) is the (optional) replacement code. This will be code that is executed (instead of the normal environment code)
when the env is executed, even though the content of the environment is hidden. The second (required) argument is the name of the environment 
which gets hidden.

\textbf{Note:} As of a (semi)recent LaTeX update, the \verb|\hideEnv| has broken, and thus anything in the testing area that 
relates to testing \verb|\hideEnv| has been commented out until it can be fixed.

\section{Intended Outcome of Test}
\newcommand{\tmpMsg}{}

Below is a ``claim'' environment with the content: 
\begin{quote}
This is a claim. I am also redefining the macro tempMsg to output ``This is tempMsg 1''.
\end{quote}
Then the macro \verb|\tempMsg| is expanded. Then the \verb|\hideEnv[claim has been hidden!]{claim}| command is executed 
and afterward the following is executed inside a second claim environment:
\begin{quote}
This is a claim. I am also redefining the macro tempMsg to output ``This is tempMsg 2''.
\end{quote}
Then the macro \verb|\tempMsg| is expanded a second time.

Then the macro \verb|\clearEnv{claim}| is executed, followed by 
\verb|\newenvironment{claim}{claim has been cleared and then defined as a new environment!}{}| to demonstrate that no error occurs, 
i.e. that the clearEnv command has undefined the claim. 

Finally, we have
\begin{quote}
So everything still works!
\end{quote}
in a claim environment - to demonstrate that it still works and to show the new environment content has been successfully applied.

\section{Start of Test/Demo Area}

\begin{claim}
This is a claim. I am also redefining the macro tempMsg to output ``This is tempMsg 1''. \renewcommand{\tmpMsg}{This is tempMsg 1}
\end{claim}

%\tmpMsg

%\hideEnv[claim has been hidden!]{problem}

%\begin{problem}
%    This is a claim. I am also redefining the macro tempMsg to output ``This is tempMsg 2''. \renewcommand{\tmpMsg}{This is tempMsg 2}
%\end{problem}

\tmpMsg

\clearEnv{claim}

\newenvironment{claim}{claim has been cleared and then defined as a new environment!}{}

\begin{claim}
So everything still works!
\end{claim}


\hrulefill


\end{document}