\documentclass{ximera}
\title{Extra graph test}


\begin{document}
\begin{abstract}
    Tests/Demo for the graph command.
\end{abstract}
\maketitle

%% Default \begin{tikzpicture} options
\tikzset{
 xmGraphTikz/.style={},
 panel/.style={},
%  xmDesmosWidth/.code 2 args={\def\xmDesmosWidth{#2}}, % sets a macro when applied
%  xmDesmosWidth/.default=600,    % pixels
%  xmDesmosHeight/.code 2 args={\def\xmDesmosHeight{#2}}, % sets a macro when applied
%  xmDesmosHeight/.default=300,    % pixels
 }

 %% Default \begin{axis} options
\pgfplotsset{
 xmGraphPlotAxis/.style={
  xlabel={$x$},  % x-axis label
  ylabel={$y$},  % y-axis label	
	axis lines=middle,
	axis equal image,
	axis line style={->},
%   panel,
 },
  panel/.code={\def\xmGraphShowPanel{true}}, % sets a macro when applied
  % xmGraphShowDesmos/.code={\def\xmGraphShowDesmos{true}}, % sets a macro when applied
  % % xmGraphShowDesmos/.default={true}, 
  % xmGraphNoDesmos/.code={\let\xmGraphShowDesmos\undefined}, 
  % % xmGraphShowDesmos,
  xmGraphShowCommand/.code={\def\xmGraphShowCommand{true}}, % sets a macro when applied
  % xmGraphShowTikZ/.code={\def\xmGraphShowTikZ{true}}, % sets a macro when applied
  % xmGraphNoTikZ/.code={\let\xmGraphShowTikZ\undDefInEd}, % sets a macro when applied
  xmDesmosWidth/.store in=\xmDesmosWidth,
  xmDesmosWidth/.default=600px,
  xmDesmosHeight/.store in=\xmDesmosHeight,
  xmDesmosHeight/.default=300px,
}

% \pgfkeys{
%   /pgfplots/xmGraphShowDesmos/.store in=\xmGraphShowDesmos,
%   /pgfplots/xmGraphShowDesmos/.initial=false,   % initial default
%   /pgfplots/xmGraphShowDesmos/.default=true,    % if used as 'xmGraphShowDesmos' -> true
%   /pgfplots/xmGraphNoDesmos/.code={% explicit "off"
%      \pgfkeyssetvalue{/pgfplots/xmGraphShowDesmos}{false}%
%   },
% }

\def\xmGraphShowDesmos{true}  % default NOT inside the style: DOES NOT WORK

% \pgfplotsset{
%   % plot-level style
%   xmGraphPlot/.style={
%     /pgfplots/.cd, % <- ensures keys are interpreted as pgfplots keys
%     domain=-2:2,
%     samples=50,
%     thick,
%   },
%   % axis-level style
%   xmGraphPlotAxis/.style={
%     /pgfplots/.cd,
%     xmin=-2.3, xmax=2.3,
%     ymin=-1, ymax=4,
%     axis equal image,
%     axis lines=middle,
%     axis line style=->,
%   }
% }

\ExplSyntaxOn

% Helper function to process each item
%%% The 'multi'-part, with f(x),g(x,f(x), does NOT yet work )
\cs_new_protected:Nn \__multiplot_get_function:n {
  % Check if the item contains slashes (legend and/or options separator)
  % IN 2025 ? \seq_set_split_with_nested:nnnN { / } { | | } { #1 } \l_tmpa_seq
  \seq_set_split:Nnn \l_tmpa_seq { // } { #1 }
  \int_case:nn { \seq_count:N \l_tmpa_seq } {
    { 1 } {
      % No legend, no options: just plot function
         #1
    }
    { 2 } {
      \seq_pop_left:NN \l_tmpa_seq \l_tmpa_tl  % function
      \l_tmpa_tl
    }
    { 3 } {
      \seq_pop_left:NN \l_tmpa_seq \l_tmpa_tl  % function
      \l_tmpa_tl
    }
  }
}

\cs_new_protected:Nn \__multiplot_process:nn {
  % Check if the item contains slashes (legend and/or options separator)
  % IN 2025 ? \seq_set_split_with_nested:nnnN { / } { | | } { #1 } \l_tmpa_seq
  \seq_set_split:Nnn \l_tmpa_seq { // } { #2 }
  \int_case:nn { \seq_count:N \l_tmpa_seq } {
    { 1 } {
      % No legend, no options: just plot function
      \addplot[xmGraphPlot, #1] { #2 };
    }
    { 2 } {
      % Has legend: plot function and add legend entry
      \seq_pop_left:NN \l_tmpa_seq \l_tmpa_tl  % function
      \seq_pop_left:NN \l_tmpa_seq \l_tmpb_tl  % legend
      \addplot[xmGraphPlot, #1] { \l_tmpa_tl };
      % Only add legend if not empty
      %\tl_if_empty:NF \l_tmpb_tl {
        \addlegendentry{\l_tmpb_tl}
      %}
    }
    { 3 } {
      % Has legend and options: plot function with custom options
      \seq_pop_left:NN \l_tmpa_seq \l_tmpa_tl  % function
      \seq_pop_left:NN \l_tmpa_seq \l_tmpb_tl  % legend
      \seq_pop_left:NN \l_tmpa_seq \l_tmpc_tl  % options
      \addplot[xmGraphPlot, #1,  \l_tmpc_tl] { \l_tmpa_tl };
      % Only add legend if not empty
      \tl_if_empty:NF \l_tmpb_tl {
        \addlegendentry{\l_tmpb_tl}
        }
    }
  }
}

\cs_new_protected:Nn \xmultiplot:nn {
  \clist_map_inline:nn { #2 } {
    \__multiplot_process:nn { #1 } { ##1 }
  }
}

\NewDocumentCommand \xmGraphPlot { o m } { \xmultiplot:nn {#1} {#2} }
\NewDocumentCommand \xmGraphGetFunction { m } { \__multiplot_get_function:n {#1} }
\ProvideDocumentCommand \xmGraphLegend { m } {}
\ProvideDocumentCommand \xmGraphExtra { m } {}

\ExplSyntaxOff

\newcommand\GetKeyFromStyle[2]{%
  \pgfkeys{/pgfplots/.cd,#1}%
  \pgfkeysvalueof{/pgfplots/#2}%
}

\newcommand\GetPGFplotsKeyFromStyle[2]{%
  % #1 = style name
  % #2 = key name
  % Save current value (to restore if undefined)
  \pgfkeys{/pgfplots/.cd,#1}%
  \pgfkeys{/pgfplots/#1} % executes the style code

  \typeout{KEY #2 in #1}

  % Try global key first
  \pgfkeysifdefined{/pgfplots/#2}{%
    \pgfkeysvalueof{/pgfplots/#2}%
    \typeout{KEY found: #2 in #1: \pgfkeysvalueof{/pgfplots/#2}}
  }{%
    % Try axis key
    \pgfkeysifdefined{/pgfplots/axis/#2}{%
      \pgfkeysvalueof{/pgfplots/axis/#2}%
      \typeout{KEY found in axis: #2 in #1: \pgfkeysvalueof{/pgfplots/axis/#2}}
    }{%
        % Try dummy option
        \pgfkeysifdefined{/mydymmyoption/#2}{%
        \pgfkeysvalueof{/mydymmyoption/#2}%
        \typeout{KEY found dummyoption: #2 in #1: \pgfkeysvalueof{/mydymmyoption/#2}}
        }
        {
        % Key not found → return empty
        \typeout{KEY not found: #2 in #1}
        }%
    }%
  }
}

\makeatletter
\renewcommand\GetPGFplotsKeyFromStyle[2]{%
  % #1 = style name
  % #2 = key name
  \pgfkeys{/pgfplots/.cd,#1}% activate style
  %\typeout{KEY #2 in #1}
  \pgfkeysifdefined{/pgfplots/#2}{%
    \pgfkeysvalueof{/pgfplots/#2}%
   % \typeout{KEY found: #2 in #1: \pgfkeysvalueof{/pgfplots/#2}}
  }{%
    \pgfkeysifdefined{/pgfplots/axis/#2}{%
      \pgfkeysvalueof{/pgfplots/axis/#2}%
      %\typeout{KEY axis found: #2 in #1: \pgfkeysvalueof{/pgfplots/#2}}
    }{%
    %\pgfkeysifdefined{/pgfplots/.@cmd/#2}{TRUE}{FALSE}%
      % undefined → return empty
    }%
  }%
}
\makeatother

\makeatletter
% #1 = macro to store value
% #2 = style name
% #3 = key name
\newcommand\GetPGFplotsKeyFromStyleToMacro[3]{%
  %\begingroup
    \def#1{}% initialize macro to empty
    \pgfkeys{/pgfplots/.cd,#2}% activate the style
    % Try normal key
    \pgfkeysifdefined{/pgfplots/#3}{%
      \pgfkeysgetvalue{/pgfplots/#3}{#1}%
    }{%
      % Try axis key
      \pgfkeysifdefined{/pgfplots/axis/#3}{%
        \pgfkeysgetvalue{/pgfplots/axis/#3}{#1}%
      }{%
        % Key not found → leave empty
        \def#1{}%
      }%
    }%
  %\endgroup
}

\makeatother

\renewcommand{\graph}[2][]{
  \begingroup
  \ifdefined\xmGraphShowDesmos    % generic tikz-options confuse the pgfkeys ...
  %\typeout{SETTING tmpstyle (\xmGraphShowDesmos\  for #1)}
\pgfplotsset{   %% The #1 will process the additional options, so that \xmGraphShowTikZ works !!!
  tmpstyle/.style={
    /pgfplots/xmGraphPlot,        % apply plot keys
    /pgfplots/xmGraphPlotAxis,    % apply axis keys
 %   /pgfplots/.cd,                % ensure any extra keys are interpreted in pgfplots
    #1                            % allow overrides
  }
}
  \pgfkeys{/pgfplots/.cd,#1}% activate style
\else
  %\typeout{SKIPPING tmpstyle for #1}
\fi

\ifdefined\HCode
  \HCode{<div class="xmdesmos">}
\else
  \def\xmGraphShowTikZ{true}  %% ALWAYS show TikZ in PDF !!!
\fi

  \ifdefined\xmGraphShowCommand
  \ifdefined\HCode
   % \HCode{<h1> Command: \string\graph[#1]{#2}  </h1> }
  \else
   %  {\Large\bf With options #1 }
  \fi
  \fi



\ifdefined\HCode\ifdefined\xmGraphShowDesmos
% \pgfplotsset{
%   xxxxtmpstyle/.style = { 
%     /pgfplots/.append style={xmGraphPlot}  % apply xmGraphPlot in axis context
%     /pgfplots/axis/.append style={xmGraphPlotAxis}  % apply xmGraphPlot in axis context
%     /pgfplots/axis/.append style={#1}  % apply xmGraphPlot in axis context
% },
%  }



% \GetPGFplotsKeyFromStyleToMacro{\myval}{tmpstyle}{axis equal image}
% axis equal image = \myval % prints "-2:2"

% \GetPGFplotsKeyFromStyleToMacro{\myval}{tmpstyle}{xlabel}
% xlabel = \myval % prints "-2:2"
% xlabel = \HCode{\myval} % prints "-2:2"

%\typeout{AXIS KEY: \GetPGFplotsKeyFromStyle{tmpstyle}{axis equal image}}
% AXIS : \GetPGFplotsKeyFromStyle{tmpstyle}{axis equal image}
% PANEL : \GetPGFplotsKeyFromStyle{tmpstyle}{panel}
% XLABEL : \GetPGFplotsKeyFromStyle{tmpstyle}{xlabel}

 \HCode{<div class="desmos-placeholder" data-options="}
    %%% still an issue with $x$ vs x   (Desmos prints the $'s !)
    % \GetPGFplotsKeyFromStyleToMacro{\myval}{tmpstyle}{xlabel}
    % xlabel=\HCode{\myval},
    % \GetPGFplotsKeyFromStyleToMacro{\myval}{tmpstyle}{ylabel}
    % ylabel=\HCode{\myval},
    % ylabel=\HCode{\GetPGFplotsKeyFromStyle{tmpstyle}{ylabel}},
    xmax=\GetPGFplotsKeyFromStyle{tmpstyle}{xmax},
    xmin=\GetPGFplotsKeyFromStyle{tmpstyle}{xmin},
    ymin=\GetPGFplotsKeyFromStyle{tmpstyle}{ymin},
    ymax=\GetPGFplotsKeyFromStyle{tmpstyle}{ymax},
    \ifdefined\xmGraphShowPanel    panel=true, \fi
    \ifdefined\xmDesmosWidth       xmDesmosWidth=\xmDesmosWidth, \fi
    \ifdefined\xmDesmosHeight      xmDesmosHeight=\xmDesmosHeight, \fi
    \if\relax\detokenize\expandafter{\GetPGFplotsKeyFromStyle{tmpstyle}{axis equal image}}\relax
      axis equal image=true,
    \fi
 \HCode{" data-graph='}\detokenize{#2}\HCode{'"></div>}%
\fi\fi%
\ifdefined\xmGraphShowTikZ%
  \begin{tikzpicture}[xmGraphTikz]
  \begin{axis}[xmGraphPlotAxis,#1] 
    \xmGraphPlot[]{#2} 
    % \xmGraphPlot[#1]{#2} 
    \xmGraphLegend{}
    \xmGraphExtra{}
  \end{axis}
  \end{tikzpicture}
\fi

\endgroup
}

\ExplSyntaxOn
\bool_set_true:N \l__trace_errors_bool
\ExplSyntaxOff

\NewDocumentCommand{\xmgraph}{ O{} m }
{
  % #1 = optional key-value options
  % #2 = the math expression (can include \frac, \sqrt, etc.)
  
  % Store argument verbatim
  \tl_set:Nn \l_tmpa_tl {#2}

  % Expand token list at use, safely passing to \graph
  \exp_args:No \graph [#1] {\l_tmpa_tl}
}

\ExplSyntaxOff

% \newcommand{\xmgraph}[2][]{$\graph{#2}$}
\pgfplotsset{
 xmGraphPlot/.append style={
  domain=-2:2,
 },
  % Axis-only keys go in a separate style
  xmGraphPlotAxis/.append style={
    xmin=-2.3,
    xmax=2.3,
    ymin=-1, ymax=4,
    % xmDesmosWidth=500px,
    % xmDesmosHeight=500px,
    % ymin, ymax, tick settings etc.
  }   
 }


%  xmax=\GetPGFplotsKeyFromStyle{xmGraphPlot}{xmax}

%  xmax=\GetPGFplotsKeyFromStyle{xmGraphPlotAxis}{xmax}
%  \graph{x^2/2//function//blue}


\def\xmGraphShowCommand{true}

  \graph[xmin=-7, xmax=5, ymin=-5, ymax=5]{\frac{4}{x+2},x=-2}


  \pgfplotsset{
 xmGraphPlot/.append style={
  domain=-2:2,
 },
  % Axis-only keys go in a separate style
  xmGraphPlotAxis/.append style={
    xmin=-2.3,
    xmax=2.3,
    ymin=-1, ymax=4,
    xmDesmosWidth =200px,
    xmDesmosHeight=200px,
    % ymin, ymax, tick settings etc.
  }   
 }

 \graph[panel,xmin=-2.5, xmax=2.5,ymin=-0.5,ymax=8]{\frac{1}{x^2}}

  \graph[panel]{1/x^2, 1/x^4, 1/x^6, (1,1), (-1,1),x=2}
 
 
  \graph[panel,xmin=-2.5, xmax=2.5,ymin=-0.5,ymax=8]{\frac{1}{x^2}, \frac{1}{x^4}, \frac{1}{x^6}, (1,1), (-1,1)}
 
 \end{document}
