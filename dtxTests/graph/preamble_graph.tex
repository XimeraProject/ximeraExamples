
%% Default \begin{tikzpicture} options
\tikzset{
 xmGraphTikz/.style={},
 panel/.style={},
 }

 %% Default \begin{axis} options
\pgfplotsset{
 xmGraphPlot/.style={},
 xmGraphPlotAxis/.style={
  xlabel={$x$},  % x-axis label
  ylabel={$y$},  % y-axis label	
	axis lines=middle,
	axis equal image,
	axis line style={->},
%   panel,
 },
  panel/.code={\def\xmGraphShowPanel{true}}, % sets a macro when applied
  % xmGraphShowDesmos/.code={\def\xmGraphShowDesmos{true}}, % sets a macro when applied
  % % xmGraphShowDesmos/.default={true}, 
  % xmGraphNoDesmos/.code={\let\xmGraphShowDesmos\undefined}, 
  % % xmGraphShowDesmos,
  xmGraphShowCommand/.code={\def\xmGraphShowCommand{true}}, % sets a macro when applied
  % xmGraphShowTikZ/.code={\def\xmGraphShowTikZ{true}}, % sets a macro when applied
  % xmGraphNoTikZ/.code={\let\xmGraphShowTikZ\undDefInEd}, % sets a macro when applied
  xmDesmosWidth/.store in=\xmDesmosWidth,
  xmDesmosWidth/.default=600px,
  xmDesmosHeight/.store in=\xmDesmosHeight,
  xmDesmosHeight/.default=300px,
}

\def\xmGraphShowDesmos{true}  % default NOT inside the style: DOES NOT WORK

\ExplSyntaxOn
% Helper function to process each item
%%% The 'multi'-part, with f(x),g(x,f(x), does NOT yet work )
\cs_new_protected:Nn \__multiplot_get_function:n {
  % Check if the item contains slashes (legend and/or options separator)
  % IN 2025 ? \seq_set_split_with_nested:nnnN { / } { | | } { #1 } \l_tmpa_seq
  \seq_set_split:Nnn \l_tmpa_seq { // } { #1 }
  \int_case:nn { \seq_count:N \l_tmpa_seq } {
    { 1 } {
      % No legend, no options: just plot function
         #1
    }
    { 2 } {
      \seq_pop_left:NN \l_tmpa_seq \l_tmpa_tl  % function
      \l_tmpa_tl
    }
    { 3 } {
      \seq_pop_left:NN \l_tmpa_seq \l_tmpa_tl  % function
      \l_tmpa_tl
    }
  }
}

\cs_new_protected:Nn \__multiplot_process:nn {
  % Check if the item contains slashes (legend and/or options separator)
  % IN 2025 ? \seq_set_split_with_nested:nnnN { / } { | | } { #1 } \l_tmpa_seq
  \seq_set_split:Nnn \l_tmpa_seq { // } { #2 }
  \int_case:nn { \seq_count:N \l_tmpa_seq } {
    { 1 } {
      % No legend, no options: just plot function
      \addplot[xmGraphPlot, #1] { #2 };
    }
    { 2 } {
      % Has legend: plot function and add legend entry
      \seq_pop_left:NN \l_tmpa_seq \l_tmpa_tl  % function
      \seq_pop_left:NN \l_tmpa_seq \l_tmpb_tl  % legend
      \addplot[xmGraphPlot, #1] { \l_tmpa_tl };
      % Only add legend if not empty
      %\tl_if_empty:NF \l_tmpb_tl {
        \addlegendentry{\l_tmpb_tl}
      %}
    }
    { 3 } {
      % Has legend and options: plot function with custom options
      \seq_pop_left:NN \l_tmpa_seq \l_tmpa_tl  % function
      \seq_pop_left:NN \l_tmpa_seq \l_tmpb_tl  % legend
      \seq_pop_left:NN \l_tmpa_seq \l_tmpc_tl  % options
      \addplot[xmGraphPlot, #1,  \l_tmpc_tl] { \l_tmpa_tl };
      % Only add legend if not empty
      \tl_if_empty:NF \l_tmpb_tl {
        \addlegendentry{\l_tmpb_tl}
        }
    }
  }
}

\cs_new_protected:Nn \xmultiplot:nn {
  \clist_map_inline:nn { #2 } {
    \__multiplot_process:nn { #1 } { ##1 }
  }
}

\NewDocumentCommand \xmGraphPlot { o m } { \xmultiplot:nn {#1} {#2} }
\NewDocumentCommand \xmGraphGetFunction { m } { \__multiplot_get_function:n {#1} }
\ProvideDocumentCommand \xmGraphLegend { m } {}
\ProvideDocumentCommand \xmGraphExtra { m } {}

\ExplSyntaxOff

% \newcommand\GetKeyFromStyle[2]{%
%   \pgfkeys{/pgfplots/.cd,#1}%
%   \pgfkeysvalueof{/pgfplots/#2}%
% }

\makeatletter
\newcommand\GetPGFplotsKeyFromStyle[2]{%
  % #1 = style name
  % #2 = key name
  \pgfkeys{/pgfplots/.cd,#1}% activate style
  %\typeout{KEY #2 in #1}
  \pgfkeysifdefined{/pgfplots/#2}{%
    \pgfkeysvalueof{/pgfplots/#2}%
   % \typeout{KEY found: #2 in #1: \pgfkeysvalueof{/pgfplots/#2}}
  }{%
    \pgfkeysifdefined{/pgfplots/axis/#2}{%
      \pgfkeysvalueof{/pgfplots/axis/#2}%
      %\typeout{KEY axis found: #2 in #1: \pgfkeysvalueof{/pgfplots/#2}}
    }{%
    %\pgfkeysifdefined{/pgfplots/.@cmd/#2}{TRUE}{FALSE}%
      % undefined → return empty
    }%
  }%
}
\makeatother


\newcommand{\xmgraph}[2][]{
  \begingroup
  \ifdefined\xmGraphShowDesmos    % generic tikz-options confuse the pgfkeys ...
    \typeout{SETTING tmpstyle (\xmGraphShowDesmos\  for #1)}
    \pgfplotsset{   %% The #1 will process the additional options, so that \xmGraphShowTikZ works !!!
      tmpstyle/.style={
        /pgfplots/xmGraphPlot,        % apply plot keys
        /pgfplots/xmGraphPlotAxis,    % apply axis keys
        #1                            % allow overrides
      }
    }
    \pgfkeys{/pgfplots/.cd,#1}        % activate style
  \else
    \typeout{SKIPPING tmpstyle for #1}
  \fi

  \ifdefined\HCode
    \HCode{<div class="xmdesmos">}
  \else
    \def\xmGraphShowTikZ{true}  %% ALWAYS show TikZ in PDF !!!
  \fi

  % Debugging/convenience : Automatically add the command to the output
  \ifdefined\xmGraphShowCommand
  \ifdefined\HCode
    \HCode{<h1> Command: \string\graph[#1] for  }\detokenize{#2} \HCode{</h1> }
  \else
     {\Large\bf With options #1 }
  \fi
  \fi

  % Do the DESMOS part
  \ifdefined\HCode\ifdefined\xmGraphShowDesmos
    \HCode{<div class="desmos-placeholder" data-options="}
        %%% still an issue with $x$ vs x   (Desmos prints the $'s !)
        % \def\myval{\GetPGFplotsKeyFromStyle}tmpstyle}{xlabel}
        % xlabel=\HCode{\myval},
        xmax=\GetPGFplotsKeyFromStyle{tmpstyle}{xmax},
        xmin=\GetPGFplotsKeyFromStyle{tmpstyle}{xmin},
        ymin=\GetPGFplotsKeyFromStyle{tmpstyle}{ymin},
        ymax=\GetPGFplotsKeyFromStyle{tmpstyle}{ymax},
        \ifdefined\xmGraphShowPanel    panel=true, \fi
        \ifdefined\xmDesmosWidth       xmDesmosWidth=\xmDesmosWidth, \fi
        \ifdefined\xmDesmosHeight      xmDesmosHeight=\xmDesmosHeight, \fi
        \if\relax\detokenize\expandafter{\GetPGFplotsKeyFromStyle{tmpstyle}{axis equal image}}\relax
          axis equal image=true,
        \fi
    \HCode{" data-graph='}\detokenize{#2}\HCode{'"></div>}%
  \fi\fi%
  % Do the TIKZ part
  \ifdefined\xmGraphShowTikZ%
    \begin{tikzpicture}[xmGraphTikz]
    \begin{axis}[xmGraphPlotAxis,#1] 
      \xmGraphPlot[]{#2} 
      % \xmGraphPlot[#1]{#2} 
      \xmGraphLegend{}
      \xmGraphExtra{}
    \end{axis}
    \end{tikzpicture}
  \fi
\endgroup
}

% \ExplSyntaxOn
% \bool_set_true:N \l__trace_errors_bool
% \ExplSyntaxOff


\ifdefined\xmGraphNoRedefine    % This would give the ximera.4ht version
\else
  \AtBeginDocument{\let\graph\xmgraph}
\fi