\documentclass{ximera}
\title{Expandables}

\begin{document}
\begin{abstract}
\end{abstract}
\maketitle

{{\Huge \bfseries Last Updated: \today}} \\

\section{Basic Usage}

The environment expandable creates an expandable section in the html, so that the content can be shown online, but is initially loaded as hidden.

This is a purely html based feature currently.

\section{Intended Outcome of Test}

There should be two html segments below. 

The first one should have \texttt{A first example of an expandable environment.} followed by 
a collapsed box with an arrow to the right that allows you to expand it and show: 
\begin{quote}
This is the content of the expandable environment. It can contain text, images, or any other LaTeX content.
\end{quote}

Below that should have \texttt{Another example of an expandable environment, now a remark.} followed by 
a collapsed box with an arrow to the right that allows you to expand it and show: 
\begin{quote}
This is the content of the expandable environment. It can contain text, images, or any other LaTeX content.
\end{quote}


\section{Start of Test/Demo Area}

\begin{expandable}{example}{A first example of an expandable environment.}
This is the content of the expandable environment. It can contain text, images, or any other LaTeX content.
\end{expandable}

\begin{expandable}{remark}{Another example of an expandable environment, now a remark.}
This is the content of the expandable environment. It can contain text, images, or any other LaTeX content.
\end{expandable}



\hrulefill
\end{document}
