\documentclass{ximera}
\title{Validator Activity 2}

%\typeout{Start loading xmPreamble.tex}%


\newcommand{\ximera}{Ximera}
\usepackage{lipsum}
% Add here extra macro's that are loaded automatically by all documents of claas 'ximera' or 'xourse' in this repo

%%
%%  Example:
%%
% \newcommand{\R}{\mathbb{R}
\begin{document}
\begin{abstract}
    Another testbed file for Validators
\end{abstract}
\maketitle

{{\Huge \bfseries Last Updated: \today}} \\

\section{Basic Usage}

This file demonstrates the following validators: factorCheck
This is implemented using the \verb|validator=factorCheck| as an optional parameter in the \verb|\answer| command.

\section{Intended Outcome of Test}
    \subsection*{factorCheck}
        This validator checks to see if the provided ``factored form'' from the student 
        is actually factored in a similar way to the author-provided ``factored form''.
        
        In the test/demo area are a number of problems that should demonstrate the factoring. The answers should be implied by the prompt, 
        but the answers (in order) are: $(x^2-4)(x-4)$, $(x-2)(x+2)(x-4)$, $(x-1)^3(x+1)(x^2-1)$

\section{Start of Test/Demo Area}

    \subsection*{factorCheck}
        \begin{problem}
            Consider the polynomial $x^3 - 4x^2 - 4x + 16$. You might want the student to just do the factor by grouping step 
            and want them to enter in $(x^2-4)(x-4)$. 
            Try trying in the full polynomial versus a fully factored version, 
            versus the desired version: $\answer[validator=factorCheck]{(x^2-4)(x-4)}$
        \end{problem}

        \begin{problem}
            Now, let's say you want them to fully factor, not just factor by grouping. 
            Try the factor by grouping version versus unfactored vs fully factored here: $\answer[validator=factorCheck]{(x-2)(x+2)(x-4)}$
        \end{problem}
        
        \begin{problem}
            Adding another problem here that deliberately has some repeated factors to show they work. 
            The answer should be $(x-1)^3(x+1)(x^2-1)$.
            
            $\answer[validator=factorCheck]{(x-1)^3(x+1)(x^2-1)}$
        \end{problem}


\hrulefill
\end{document}