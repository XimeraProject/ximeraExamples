\documentclass{ximera}
\title{Theorem-Style Environments Basic Usage}

%\typeout{Start loading xmPreamble.tex}%


\newcommand{\ximera}{Ximera}
\usepackage{lipsum}
% Add here extra macro's that are loaded automatically by all documents of claas 'ximera' or 'xourse' in this repo

%%
%%  Example:
%%
% \newcommand{\R}{\mathbb{R}% Only add a preamble file if it is actually necessary for the demo/test.
\begin{document}
\begin{abstract}
    Tests/Examples of basic usage of theorem-like environments.
\end{abstract}
\maketitle

{{\Huge \bfseries Last Updated: \today}} \\

\section{Basic Usage}
This demonstrates and helps test all the (already implemented) theorem-like environments.

\section{Intended Outcome of Test}
Below should be a (long) list of (mostly empty) of environments that should stylize as theorem-like environments.

\section{Start of Test/Demo Area}
\begin{algorithm}
    This is an algorithm Environment.
\end{algorithm}% Used in theorems.dtx

\begin{axiom}
    This is an axiom Environment.
\end{axiom}% Used in theorems.dtx

\begin{claim}
    This is a claim Environment.
\end{claim}% Used in theorems.dtx

\begin{conclusion}
    This is a conclusion Environment.
\end{conclusion}% Used in theorems.dtx

\begin{condition}
    This is a condition Environment.
\end{condition}% Used in theorems.dtx

\begin{conjecture}
    This is a conjecture Environment.
\end{conjecture}% Used in theorems.dtx

\begin{corollary}
    This is a corollary Environment.
\end{corollary}% Used in theorems.dtx

\begin{criterion}
    This is a criterion Environment.
\end{criterion}% Used in theorems.dtx

\begin{definition}
    This is a definition Environment.
\end{definition}% Used in theorems.dtx

\begin{example}
    This is an example Environment.
\end{example}% Used in theorems.dtx

\begin{explanation}
    This is an explanation Environment.
\end{explanation}% Used in theorems.dtx

\begin{fact}
    This is a fact Environment.
\end{fact}% Used in theorems.dtx

\begin{lemma}
    This is a lemma Environment.
\end{lemma}% Used in theorems.dtx

\begin{formula}
    This is a formula Environment.
\end{formula}% Used in theorems.dtx

\begin{idea}
    This is an idea Environment.
\end{idea}% Used in theorems.dtx

\begin{notation}
    This is a notation Environment.
\end{notation}% Used in theorems.dtx

\begin{model}
    This is a model Environment.
\end{model}% Used in theorems.dtx

\begin{observation}
    This is an observation Environment.
\end{observation}% Used in theorems.dtx

\begin{proposition}
    This is a proposition Environment.
\end{proposition}% Used in theorems.dtx

\begin{paradox}
    This is a paradox Environment.
\end{paradox}% Used in theorems.dtx

\begin{procedure}
    This is a procedure Environment.
\end{procedure}% Used in theorems.dtx

\begin{remark}
    This is a remark Environment.
\end{remark}% Used in theorems.dtx

\begin{summary}
    This is a summary Environment.
\end{summary}% Used in theorems.dtx

\begin{template}
    This is a template Environment.
\end{template}% Used in theorems.dtx

\begin{theorem}
    This is a theorem Environment.
\end{theorem}% Used in theorems.dtx

\begin{warning}
    This is a warning Environment.
\end{warning}% Used in theorems.dtx

\hrulefill


\end{document}
