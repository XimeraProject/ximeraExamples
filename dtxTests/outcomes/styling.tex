\documentclass{ximera}
\title{Outcome Style Test Activity}
\date{Last Updated: 6/5/2025}

% Stylize the outcome block.
\renewcommand{\preOutcomeLine}{This marks the beginning of the outcome line. }
\renewcommand{\postOutcomeLine}{ This marks the end of Outcome Line.\\}
\renewcommand{\preOutcomeBlock}{This is the beginning of a boring block of outcomes. No itemize.\\}
\renewcommand{\postOutcomeBlock}{End of Block... a boring block without any itemize.\\}
\renewcommand{\outcomeHeader}{This is the header - so you should see a block below this.\\}


% List of outcomes that should display where "\displayOutcomes" is used.
\outcome{Answer some questions about some stuff.}
\outcome{Do something productive.}
\outcome{Learn how Questions work maybe.}
\outcome{Maybe other stuff - who knows!}
%\typeout{Start loading xmPreamble.tex}%


\newcommand{\ximera}{Ximera}
\usepackage{lipsum}
% Add here extra macro's that are loaded automatically by all documents of claas 'ximera' or 'xourse' in this repo

%%
%%  Example:
%%
% \newcommand{\R}{\mathbb{R}

\begin{document}
\begin{abstract}
    A testbed file for Outcomes.
\end{abstract}
\maketitle

\section{Intended Outcome of Test}

In the preamble, before we use the outcome commands, we change the styling of the outcomes using the following commands:
\begin{description}
    \item[preOutcomeLine] The command \verb|\preOutcomeLine| is a macro that holds the LaTeX code that is prepended to each \verb|\outcome{text}| text. 
            This is intended to be used by content authors to stylize their outcome items.\\
            By default, the content of \verb|\preOutcomeLine| is: \verb|\item |
    \item[postOutcomeLine] The command \verb|\postOutcomeLine| is a macro that holds the LaTeX code that is appended to each \verb|\outcome{text}| text. 
            This is intended to be used by content authors to stylize their outcome items.\\
            By default, the content of \verb|\postOutcomeLine| is empty
    \item[preOutcomeBlock] The command \verb|\preOutcomeBlock| is a macro that holds the LaTeX code that is executed prior to any of the \verb|\outcome| commands. 
            This is intended to be used by content authors to stylize their outcome lists.\\
            By default, the content of \verb|\preOutcomeBlock| is: \\
            \verb|At the end of this section, students should be able to... \begin{itemize}|
    \item[postOutcomeBlock] The command \verb|\postOutcomeBlock| is a macro that holds the LaTeX code that is executed after all of the \verb|\outcome| commands are llisted. 
            This is intended to be used by content authors to stylize their outcome lists.\\
            By default, the content of \verb|\postOutcomeBlock| is: \verb|\end{itemize} So go forth and learn!|
    \item[outcomeHeader] The command \verb|\outcomeHeader| is a macro that holds the LaTeX code that is displayed prior to the actual outcome block. 
            This is intended to be used by content authors to display a header - e.g. for an expandable or hover text.\\
            By default, the content of \verb|\outcomeHeader| is: \verb|Goals for this Section (Hover over me to see!)|
    \item[outcomeBlock] The command \verb|\outcomeBlock| is an html only macro that provides the html code to support the \verb|\outcomeHeader| macro content. 
        This should largely be left alone.
\end{description}

We have changed the styling from the above defaults using the following commands:
\begin{itemize}
    \item \verb|\preOutcomeLine{This marks the beginning of the outcome line. } |
    \item \verb|\postOutcomeLine{ This marks the end of Outcome Line.\\} |
    \item \verb|\preOutcomeBlock{This is the beginning of a boring block of outcomes. No itemize.\\} |
    \item \verb|\postOutcomeBlock{End of Block... a boring block without any itemize.\\} |
    \item \verb|\outcomeHeader{This is the header - so you should see a block below this.\\} |
\end{itemize}

Then, again in the preamble, we used the following commands to populate the outcomes:
\begin{itemize}
    \item \verb|\outcome{Answer some questions about some stuff.} |
    \item \verb|\outcome{Do something productive.} |
    \item \verb|\outcome{Learn how Questions work maybe.} |
    \item \verb|\outcome{Maybe other stuff - who knows!} |
\end{itemize}

The test area should look (approximately) like the following:


This is the header - so you should see a block below this.\\
This is the beginning of a boring block of outcomes. No itemize.\\
This marks the beginning of the outcome line. Answer some questions about some stuff. This marks the end of Outcome Line.\\
This marks the beginning of the outcome line. Do something productive. This marks the end of Outcome Line.\\
This marks the beginning of the outcome line. Learn how Questions work maybe. This marks the end of Outcome Line.\\
This marks the beginning of the outcome line. Maybe other stuff - who knows! This marks the end of Outcome Line.\\
End of Block... a boring block without any itemize.




\section{Start of Test/Demo Area}

\displayOutcomes

\section{End of Test/Demo Area}

Some ending text to ensure that there is no unintended output between end of the test/demo and the end of the document.

\end{document}