\documentclass{ximera}
\title{TMPAct}

\begin{document}
\begin{abstract}
Temp activity for testing.
\end{abstract}
\maketitle

This is some text before javascript environment is loaded.

xxx

\begin{javascript}
// NOTE: The below are intended to be used inside an \answer optional argument with the validator key, NOT in a validator environment.

// sameDerivative checks to see if the derivative with respect to x and c are equal.
// Because of how they are loaded, I need to currently manually check each variable letter, so we do a full comparison
// Currently not checking e, f, g, h, or i as variables because they are special reserve letters in math.
  sameDerivative = function(a,b) {
    return (
    a.derivative('a').equals( b.derivative('a') ) &&
    a.derivative('b').equals( b.derivative('b') ) &&
    a.derivative('c').equals( b.derivative('c') ) &&
    a.derivative('d').equals( b.derivative('d') ) &&
    a.derivative('j').equals( b.derivative('j') ) &&
    a.derivative('k').equals( b.derivative('k') ) &&
    a.derivative('l').equals( b.derivative('l') ) &&
    a.derivative('m').equals( b.derivative('m') ) &&
    a.derivative('n').equals( b.derivative('n') ) &&
    a.derivative('o').equals( b.derivative('o') ) &&
    a.derivative('p').equals( b.derivative('p') ) &&
    a.derivative('q').equals( b.derivative('q') ) &&
    a.derivative('r').equals( b.derivative('r') ) &&
    a.derivative('s').equals( b.derivative('s') ) &&
    a.derivative('t').equals( b.derivative('t') ) &&
    a.derivative('u').equals( b.derivative('u') ) &&
    a.derivative('v').equals( b.derivative('v') ) &&
    a.derivative('w').equals( b.derivative('w') ) &&
    a.derivative('x').equals( b.derivative('x') ) && 
    a.derivative('y').equals( b.derivative('y') ) &&
    a.derivative('z').equals( b.derivative('z') ) &&
    a.derivative('A').equals( b.derivative('A') ) &&
    a.derivative('B').equals( b.derivative('B') ) &&
    a.derivative('C').equals( b.derivative('C') ) &&
    a.derivative('D').equals( b.derivative('D') ) &&
    a.derivative('J').equals( b.derivative('J') ) &&
    a.derivative('K').equals( b.derivative('K') ) &&
    a.derivative('L').equals( b.derivative('L') ) &&
    a.derivative('M').equals( b.derivative('M') ) &&
    a.derivative('N').equals( b.derivative('N') ) &&
    a.derivative('O').equals( b.derivative('O') ) &&
    a.derivative('P').equals( b.derivative('P') ) &&
    a.derivative('Q').equals( b.derivative('Q') ) &&
    a.derivative('R').equals( b.derivative('R') ) &&
    a.derivative('S').equals( b.derivative('S') ) &&
    a.derivative('T').equals( b.derivative('T') ) &&
    a.derivative('U').equals( b.derivative('U') ) &&
    a.derivative('V').equals( b.derivative('V') ) &&
    a.derivative('W').equals( b.derivative('W') ) &&
    a.derivative('X').equals( b.derivative('X') ) && 
    a.derivative('Y').equals( b.derivative('Y') ) &&
    a.derivative('Z').equals( b.derivative('Z') ) 
    )
  }
\end{javascript}

This is some text after javascript environment is loaded.

\begin{problem}
    This problem shows the 'same derivative' validator. It's intended to be used to test the result of an indefinite integral, so it requires the ``$+C$'' at the end, and (should) mark any answer wrong that doesn't have it. Note that the $C$ is case-sensitive. \\
            
    Enter in any answer whose derivative is $x^2 + \sin(x) - 3$ (and don't forget the $+C$, but notice you can also add random constants to it too).
    \[
        \int x^2 + \sin(x) - 3 dx = \answer[validator=sameDerivative]{\frac{1}{3}x^3 - \cos(x) - 3x + C}
    \]
            
    Sidenote: You can avoid the problem of the $+C$ if you don't put it in the expected answer box, which will then (correctly) mark an answer wrong that \textit{does} include a $+C$. So in a sense you can force the student to include it, or not, with this validator. Although only checking the derivatives to match has obvious issues for non indefinite integrals.
\end{problem}



\end{document}