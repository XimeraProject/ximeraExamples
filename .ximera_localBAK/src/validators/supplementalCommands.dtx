% \subsubsection{debugCommands}
%   This initialized a number of debug/verbose functions to be used during the validation process for custom validators.
%       Currently we have two main debug commands, debugText and debugArray.
%       debugText takes in a string and displays that string when debugInfo = true.
%       debugArray takes in an array and displays that array when debugInfo = true.
%
%    \begin{macrocode}
%<*classXimera>
%% Apend code via \appendtoverbtoks

\appendtoverbtoks?
var debugInfo = false;

function debugText(debugMsg) {
    if (debugInfo) {
    console.log(debugMsg);
    }
}

function debugArray(array) {
    if (debugInfo) {
    console.log(array);
    }
}
?
%
%
%% Since this isn't a real validation command, just subroutines for debugging validators, we don't need to instantiate them via a command.
%    \end{macrocode}
% \subsubsection{compSubTree}
%   function allows us to compare two subtrees of a function tree for equality without messing up the original trees.
%   The function is: compSubTree(f, treeOne,treeTwo)
%            The input f should be the full math expressions object of a function that we will need for cloning,
%            treeOne is a subtree of a function tree that we want to compare 
%            treeTwo is another subtree of a function tree (possibly the same, or different from the function tree generating treeOne) that we want to compare 
%            This determines if f.treeOne = f.treeTwo in the Ximera sense of equality, and returns true or false (equal or not equal).
%   Importantly the scope is such that the original function trees *shouldn't* be impacted by this process.
%    \begin{macrocode}
%% Apend code via \appendtoverbtoks

\appendtoverbtoks?
function compSubTree(f, treeOne,treeTwo) {
    /* 
        We assume that the input f is the full matrix expressions object we will need for cloning,
        treeOne is a level of the tree that is used as the subtree to compare
        treeTwo is a level of another tree that is used as the subtree to compare.
        We want to see if f.treeOne = f.treeTwo in the Ximera sense of equality.
    */
    // First we duplicate the object structure so the equals command can be used without pitching a fit.
    let fTemp=Object.create(f);
    let gTemp=Object.create(f);
    
    // Now we assign the subtrees as the full tree of the objects.
    fTemp.tree = treeOne;
    gTemp.tree = treeTwo;
    
    // Now we return the validity of their equality using Ximera equality validation.
    return (fTemp.equals(gTemp))
}
?
%</classXimera>
%    \end{macrocode}
