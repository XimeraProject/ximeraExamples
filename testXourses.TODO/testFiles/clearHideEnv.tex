\documentclass{ximera}
\title{Clear and Hide Environments}

%\typeout{Start loading xmPreamble.tex}%


\newcommand{\ximera}{Ximera}
\usepackage{lipsum}
% Add here extra macro's that are loaded automatically by all documents of claas 'ximera' or 'xourse' in this repo

%%
%%  Example:
%%
% \newcommand{\R}{\mathbb{R}% Only add a preamble file if it is actually necessary for the demo/test.
\begin{document}
\begin{abstract}
    This demonstrates and tests the clearEnv and hideEnv commands.    
\end{abstract}
\maketitle

{{\Huge \bfseries Last Updated: \today}} \\



\section{Basic Usage}
The \verb|\clearEnv{#1}| command will attempt to undefine any provided environment, 
regardless of how it was defined (e.g. newenvironment or newEnviron commands).
The (required) argument (\#1) is the name of the intended environment you wish to undefine.

It will also attempt to undefine any default counters that may have been defined in that environment
(e.g. problem counters). It should also allow compiles where those environments are still called later in 
the document to compile (although the environment itself will do nothing, it won't throw an error and stop the compile).\\

The \verb|\hideEnv[#1]{#2}| macro allows you to hide an environment by name, ensuring it - and all its contents - 
are no longer visible on the pdf output. It also allows you an optional input to replace the block of content with 
a new (uniform) content if desired.
The first argument (\#1) is the (optional) replacement code. This will be code that is executed (instead of the normal environment code)
when the env is executed, even though the content of the environment is hidden. The second (required) argument is the name of the environment 
which gets hidden.

\section{Intended Outcome of Test}

{{\Huge I need to decide how I want to demo/testbed this, so leaving this while I think more.}}

\section{Start of Test/Demo Area}


\hrulefill


\end{document}