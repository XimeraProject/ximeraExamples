%\documentclass{ximera}
\title{Tables}

%
\usepackage{tabularray}

\ExplSyntaxOn

\__tblr_keys_define:nn { table/inner }
  {
    xmcolhead .code:n = \__tblr_keys_gput:nn { xmcolhead } {#1},
  }

\int_new:N \lTblrxmColHeadInt
\int_new:N \lTblrxmColDiffInt
\int_new:N \lTblrxmRowDiffInt
% \int_new:N \lTblrxmCellRowSpanInt
\int_new:N \lTblrxmColGroupCheck
\int_new:N \ColSpanEx
\tl_new:N \ColSpanEx_tl

\int_new:N \lTblrTestingInt
\prop_new:N \l__tblr_testing_prop

\cs_new:Npn \prop_get_to_int:NnN #1 #2 #3
  {
    \int_set:Nn #3 { \prop_item:Nn #1 { #2 } }
  }  

\cs_new_protected:Npn \__tblr_build_col_head_foot:
  {
    %% \lTblrColHeadInt can not be empty, so we append '+ 0'.
    \int_set:Nn \lTblrxmColHeadInt
      { \__tblr_prop_item:ne { inner } { xmcolhead } + 0 }
    \int_compare:nNnTF { \lTblrxmColHeadInt } > { 0 }
      {
        \__tblr_build_one_table:nnNN {1} { \lTblrxmColHeadInt }
          \c_true_bool \c_true_bool
      }
  }

%% #1: data name; #2: data index 1; #3: data index 2; #4: key
\cs_new:Npn \__tblr_data_test_item:nnnn #1 #2 #3 #4
  {
      \intarray_item:cn { g__tblr_#1_ \int_use:N \gTblrLevelInt _intarray }
        { \__tblr_data_key_to_int:nnnn {#1} {#2} {#3} {#4} }
      
  }

\__tblr_keys_define:nn { table/inner }
  {
    caption .code:n = \__tblr_keys_gput:nn { caption } {#1},
    summary .code:n = \__tblr_keys_gput:nn { summary } {#1}
  }


\tl_new:N \l__tblr_caption_tl
\str_new:N \l__tblr_caption_str

\tl_new:N \l__tblr_summary_tl
\str_new:N \l__tblr_summary_str

\cs_new:Npn \l__tblr_write_caption: #1
{
   \tl_set:Nn \l__tblr_caption_tl
      { \__tblr_prop_item:ne { inner } { caption }  }
   \str_set:Ne \l__tblr_caption_str
      { \tl_to_str:e \l__tblr_caption_tl }
    \tl_set:Nn \l__tblr_summary_tl
      { \__tblr_prop_item:ne { inner } { summary }  }
   \str_set:Ne \l__tblr_summary_str
      { \tl_to_str:e \l__tblr_summary_tl }
}

\int_new:N \l__tlbr_number_cols_int

\tl_new:N \l__tblr_colgroup_tl 

\ExplSyntaxOff
% Only add a preamble file if it is actually necessary for the demo/test.
\begin{document}
\begin{abstract}
\end{abstract}
\maketitle

\section{Basic Usage}

ximera

        With \verb|tabular| (and \verb|center|):
        \begin{center}
            \begin{tabular}{lr}
                x & 1 \\
                y & 2 \\
                z & 1
            \end{tabular}
        \end{center}
        
        With \verb|tabular and @{} | (and \verb|center|):
        \begin{center}
            \begin{tabular}{@{}l@{ }r@{.}}
             x & 1 \\
             y & 2 \\
             z & 1
            \end{tabular}
            \end{center}

    With \verb|tabular and p{} | (and \verb|center|):
        \begin{center}
            \begin{tabular}{|p{1cm}|p{2cm}|p{3cm}|}
             x & 1 & 2\\
             y & 2 & 3\\
             z & 1 & 4
            \end{tabular}
        \end{center}

    With \verb|array|:
        $$
        \begin{array}{l|r}
         x & 1 \\
         \hline
         y & 2 \\
         z & 1
        \end{array}
        $$
    With \verb|array and @{} and p{}|:
        $$
        \begin{array}{@{X }l@{|}p{2cm}|}
         x & 1 \\
         \hline
         y & 2 \\
         z & 1
        \end{array}
        $$

% \begin{tblr}{colspec={XXXX}, row{1}=purple, rowhead=1}
% Alpha
% & Beta & Gamma
% & Delta
% \\
% Epsilon & Zeta & Eta
% & Theta
% \\
% Iota
% & Kappa & Lambda & Mu
% \\
% Nu
% & Xi
% & Omicron & Pi
% \\
% Rho
% & Sigma & Tau
% & Upsilon \\
% Phi
% & Chi
% & Psi
% & Omega
% \\
% \end{tblr}        

\section{Intended Outcome of Test}

The intended outcome section should show what the output of the demo case looks like, without actually using the DemoCom itself.
Any inconsistencies in formating/display should be noted as well as possible here as well, but it is important to not use the 
DemoCom itself so that we have an objective ``anticipated output'' to use as a comparison to what is \textit{actually} output in
test. 

\section{Start of Test/Demo Area}

This should contain the actual test/demonstration of DemoCom, which can then be compared to the description above to look for 
unexpected behavior.

\hrulefill


\end{document}

%%%%%%%%%%%%%%%%%%%%%%%%%%%%%%%%%%%%%%%%%%%
%%%     For copy/paste into new files:  %%%
%%%%%%%%%%%%%%%%%%%%%%%%%%%%%%%%%%%%%%%%%%%

\documentclass{ximera}
\title{Template}

%
\usepackage{tabularray}

\ExplSyntaxOn

\__tblr_keys_define:nn { table/inner }
  {
    xmcolhead .code:n = \__tblr_keys_gput:nn { xmcolhead } {#1},
  }

\int_new:N \lTblrxmColHeadInt
\int_new:N \lTblrxmColDiffInt
\int_new:N \lTblrxmRowDiffInt
% \int_new:N \lTblrxmCellRowSpanInt
\int_new:N \lTblrxmColGroupCheck
\int_new:N \ColSpanEx
\tl_new:N \ColSpanEx_tl

\int_new:N \lTblrTestingInt
\prop_new:N \l__tblr_testing_prop

\cs_new:Npn \prop_get_to_int:NnN #1 #2 #3
  {
    \int_set:Nn #3 { \prop_item:Nn #1 { #2 } }
  }  

\cs_new_protected:Npn \__tblr_build_col_head_foot:
  {
    %% \lTblrColHeadInt can not be empty, so we append '+ 0'.
    \int_set:Nn \lTblrxmColHeadInt
      { \__tblr_prop_item:ne { inner } { xmcolhead } + 0 }
    \int_compare:nNnTF { \lTblrxmColHeadInt } > { 0 }
      {
        \__tblr_build_one_table:nnNN {1} { \lTblrxmColHeadInt }
          \c_true_bool \c_true_bool
      }
  }

%% #1: data name; #2: data index 1; #3: data index 2; #4: key
\cs_new:Npn \__tblr_data_test_item:nnnn #1 #2 #3 #4
  {
      \intarray_item:cn { g__tblr_#1_ \int_use:N \gTblrLevelInt _intarray }
        { \__tblr_data_key_to_int:nnnn {#1} {#2} {#3} {#4} }
      
  }

\__tblr_keys_define:nn { table/inner }
  {
    caption .code:n = \__tblr_keys_gput:nn { caption } {#1},
    summary .code:n = \__tblr_keys_gput:nn { summary } {#1}
  }


\tl_new:N \l__tblr_caption_tl
\str_new:N \l__tblr_caption_str

\tl_new:N \l__tblr_summary_tl
\str_new:N \l__tblr_summary_str

\cs_new:Npn \l__tblr_write_caption: #1
{
   \tl_set:Nn \l__tblr_caption_tl
      { \__tblr_prop_item:ne { inner } { caption }  }
   \str_set:Ne \l__tblr_caption_str
      { \tl_to_str:e \l__tblr_caption_tl }
    \tl_set:Nn \l__tblr_summary_tl
      { \__tblr_prop_item:ne { inner } { summary }  }
   \str_set:Ne \l__tblr_summary_str
      { \tl_to_str:e \l__tblr_summary_tl }
}

\int_new:N \l__tlbr_number_cols_int

\tl_new:N \l__tblr_colgroup_tl 

\ExplSyntaxOff
% Only add a preamble file if it is actually necessary for the demo/test.
\begin{document}
\begin{abstract}
    
\end{abstract}
\maketitle

{{\Huge \bfseries Last Updated: \today}} \\


\section{Basic Usage}


\section{Intended Outcome of Test}


\section{Start of Test/Demo Area}


\hrulefill


\end{document}
