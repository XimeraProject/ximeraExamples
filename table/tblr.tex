\documentclass{ximera}
\title{Tables: tblr}

%\typeout{Start loading xmPreamble.tex}%


\newcommand{\ximera}{Ximera}
\usepackage{lipsum}
% Add here extra macro's that are loaded automatically by all documents of claas 'ximera' or 'xourse' in this repo

%%
%%  Example:
%%
% \newcommand{\R}{\mathbb{R}% Only add a preamble file if it is actually necessary for the demo/test.
\begin{document}
\begin{abstract}
\end{abstract}
\maketitle

\section{Basic Usage}


\begin{tblr}{colspec={XXXX}, row{1}=purple, rowhead=1}
Alpha
& Beta & Gamma
& Delta
\\
Epsilon & Zeta & Eta
& Theta
\\
Iota
& Kappa & Lambda & Mu
\\
Nu
& Xi
& Omicron & Pi
\\
Rho
& Sigma & Tau
& Upsilon \\
Phi
& Chi
& Psi
& Omega
\\
\end{tblr}        

\section{Intended Outcome of Test}

The intended outcome section should show what the output of the demo case looks like, without actually using the DemoCom itself.
Any inconsistencies in formating/display should be noted as well as possible here as well, but it is important to not use the 
DemoCom itself so that we have an objective ``anticipated output'' to use as a comparison to what is \textit{actually} output in
test. 

\section{Start of Test/Demo Area}

This should contain the actual test/demonstration of DemoCom, which can then be compared to the description above to look for 
unexpected behavior.

\hrulefill


\end{document}

%%%%%%%%%%%%%%%%%%%%%%%%%%%%%%%%%%%%%%%%%%%
%%%     For copy/paste into new files:  %%%
%%%%%%%%%%%%%%%%%%%%%%%%%%%%%%%%%%%%%%%%%%%

\documentclass{ximera}
\title{Template}

%\typeout{Start loading xmPreamble.tex}%


\newcommand{\ximera}{Ximera}
\usepackage{lipsum}
% Add here extra macro's that are loaded automatically by all documents of claas 'ximera' or 'xourse' in this repo

%%
%%  Example:
%%
% \newcommand{\R}{\mathbb{R}% Only add a preamble file if it is actually necessary for the demo/test.
\begin{document}
\begin{abstract}
    
\end{abstract}
\maketitle

{{\Huge \bfseries Last Updated: \today}} \\


\section{Basic Usage}


\section{Intended Outcome of Test}


\section{Start of Test/Demo Area}


\hrulefill


\end{document}
