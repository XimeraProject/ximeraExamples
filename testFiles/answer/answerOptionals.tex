\documentclass{ximera}
\title{Answer Base Useage}

%
\usepackage{tabularray}

\ExplSyntaxOn

\__tblr_keys_define:nn { table/inner }
  {
    xmcolhead .code:n = \__tblr_keys_gput:nn { xmcolhead } {#1},
  }

\int_new:N \lTblrxmColHeadInt
\int_new:N \lTblrxmColDiffInt
\int_new:N \lTblrxmRowDiffInt
% \int_new:N \lTblrxmCellRowSpanInt
\int_new:N \lTblrxmColGroupCheck
\int_new:N \ColSpanEx
\tl_new:N \ColSpanEx_tl

\int_new:N \lTblrTestingInt
\prop_new:N \l__tblr_testing_prop

\cs_new:Npn \prop_get_to_int:NnN #1 #2 #3
  {
    \int_set:Nn #3 { \prop_item:Nn #1 { #2 } }
  }  

\cs_new_protected:Npn \__tblr_build_col_head_foot:
  {
    %% \lTblrColHeadInt can not be empty, so we append '+ 0'.
    \int_set:Nn \lTblrxmColHeadInt
      { \__tblr_prop_item:ne { inner } { xmcolhead } + 0 }
    \int_compare:nNnTF { \lTblrxmColHeadInt } > { 0 }
      {
        \__tblr_build_one_table:nnNN {1} { \lTblrxmColHeadInt }
          \c_true_bool \c_true_bool
      }
  }

%% #1: data name; #2: data index 1; #3: data index 2; #4: key
\cs_new:Npn \__tblr_data_test_item:nnnn #1 #2 #3 #4
  {
      \intarray_item:cn { g__tblr_#1_ \int_use:N \gTblrLevelInt _intarray }
        { \__tblr_data_key_to_int:nnnn {#1} {#2} {#3} {#4} }
      
  }

\__tblr_keys_define:nn { table/inner }
  {
    caption .code:n = \__tblr_keys_gput:nn { caption } {#1},
    summary .code:n = \__tblr_keys_gput:nn { summary } {#1}
  }


\tl_new:N \l__tblr_caption_tl
\str_new:N \l__tblr_caption_str

\tl_new:N \l__tblr_summary_tl
\str_new:N \l__tblr_summary_str

\cs_new:Npn \l__tblr_write_caption: #1
{
   \tl_set:Nn \l__tblr_caption_tl
      { \__tblr_prop_item:ne { inner } { caption }  }
   \str_set:Ne \l__tblr_caption_str
      { \tl_to_str:e \l__tblr_caption_tl }
    \tl_set:Nn \l__tblr_summary_tl
      { \__tblr_prop_item:ne { inner } { summary }  }
   \str_set:Ne \l__tblr_summary_str
      { \tl_to_str:e \l__tblr_summary_tl }
}

\int_new:N \l__tlbr_number_cols_int

\tl_new:N \l__tblr_colgroup_tl 

\ExplSyntaxOff
% Only add a preamble file if it is actually necessary for the demo/test.
\begin{document}
\begin{abstract}
    Base useage for answer command.
\end{abstract}
\maketitle

{{\Huge \bfseries Last Updated: \today}} \\


\section{Basic Usage}
The \verb|\answer[#1]{#2}| command has the following optional keyword-value arguments: tolerance, validator, id, format, given.

Each of these should be used in the form ``key=value'' and separated by a comma.

\textbf{Note:} Any special symbol or otherwise unparseable content in argument of answer will result in any submitted answer 
(including a blank answer) being marked correct.

\section{Intended Outcome of Test}
Below should be several answer boxes, which have the optional arguments (in order) of: tolerance=0.1, id=problemX, format=string, given=true.

The provided answers are (in order) $5$, $x$, $\text{problem answer}$, $3x^2 - 4$

The first answer should accept any decimal from $4.9$ to $5.1$, 
the second should save the provided response result as ``problemX'' (testable through console),
the third should only take ``problem answer'' as the correct answer (case and order sensitive), 
[\textbf{Note:} if format=string is not working, then it will take any permutation of those letters as correct],
the fourth answer will be displayed in the pdf (i.e. given=true says to print the answer in the answer box, rather than a question mark).


\section{Start of Test/Demo Area}
\begin{problem}

$\answer[tolerance=0.1]{5}$

$\answer[id=problemX]{x}$

$\answer[format=string]{problem answer}$

$\answer[given=true]{3x^2-4}$

\end{problem}


\hrulefill


\end{document}