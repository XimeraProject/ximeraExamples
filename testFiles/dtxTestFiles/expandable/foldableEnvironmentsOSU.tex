%% Use the OSU variant of foldable/expandable
%% By default a setup with accordion is used, that is not available on the OSU-server
%
\def\xmNotHintAsExpandable{1}
\def\xmNotExpandableAsAccordion{1}
\documentclass{ximera}

%%\typeout{Start loading xmPreamble.tex}%


\newcommand{\ximera}{Ximera}
\usepackage{lipsum}
% Add here extra macro's that are loaded automatically by all documents of claas 'ximera' or 'xourse' in this repo

%%
%%  Example:
%%
% \newcommand{\R}{\mathbb{R}

\title{Foldable environments (OSU-style)}

\begin{document}
\begin{abstract}
  Here we see expandable and foldable environments.
\end{abstract}
\maketitle

{{\Huge \bfseries Last Updated: \today}} \\

\section{Basic Usage}
The environment foldable creates a collapsable section in the html, so that the content can be hidden online, but is initially loaded as shown.

The environment expandable creates an expandable section in the html, so that the content can be shown online, but is initially loaded as hidden.

This is a purely html based feature currently.

NOTE: This does \textbf{NOT} work properly with KULeuven-styled server(s) (as used in Codespaces or local test-servers).


\section{Intended Outcome of Test}

There should be two html segments below. 

The first one should have \texttt{You can exand this one:} followed by 
a collapsed box with an arrow to the right that allows you to expand it and show \text{hello} along with an embedded youtube video.

Below that should text that says \texttt{And fold this one:} which should be followed by a box containing the text:
\texttt{You can fold me up!}, along with an arrow that allows the user to collapse the box to become hidden like the 
first box above.


\section{Start of Test/Demo Area}

You can expand this one:

\begin{expandable}
  Hello
 \begin{center}
   \youtube{0aQpLSu2fMs}
 \end{center}
\end{expandable}



And fold this one:

\begin{foldable}
  You can fold me up!
\end{foldable}


\hrulefill
\end{document}
