\documentclass{ximera}
\title{Answer Base Useage}

%\typeout{Start loading xmPreamble.tex}%


\newcommand{\ximera}{Ximera}
\usepackage{lipsum}
% Add here extra macro's that are loaded automatically by all documents of claas 'ximera' or 'xourse' in this repo

%%
%%  Example:
%%
% \newcommand{\R}{\mathbb{R}% Only add a preamble file if it is actually necessary for the demo/test.
\begin{document}
\begin{abstract}
    Base useage for answer command.
\end{abstract}
\maketitle

{{\Huge \bfseries Last Updated: \today}} \\


\section{Basic Usage}
The \verb|\answer[#1]{#2}| command has the following optional keyword-value arguments: tolerance, validator, id, format, given.

Each of these should be used in the form ``key=value'' and separated by a comma.

\textbf{Note:} Any special symbol or otherwise unparseable content in argument of answer will result in any submitted answer 
(including a blank answer) being marked correct.

\section{Intended Outcome of Test}
Below should be several answer boxes, which have the optional arguments (in order) of: tolerance=0.1, id=problemX, format=string, given=true.

The provided answers are (in order) $5$, $x$, $\text{problem answer}$, $3x^2 - 4$

The first answer should accept any decimal from $4.9$ to $5.1$, 
the second should save the provided response result as ``problemX'' (testable through console),
the third should only take ``problem answer'' as the correct answer (case and order sensitive), 
[\textbf{Note:} if format=string is not working, then it will take any permutation of those letters as correct],
the fourth answer will be displayed in the pdf (i.e. given=true says to print the answer in the answer box, rather than a question mark).


\section{Start of Test/Demo Area}
\begin{problem}

$\answer[tolerance=0.1]{5}$

$\answer[id=problemX]{x}$

$\answer[format=string]{problem answer}$

$\answer[given=true]{3x^2-4}$

\end{problem}


\hrulefill


\end{document}