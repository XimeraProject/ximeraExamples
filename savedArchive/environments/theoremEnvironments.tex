%\documentclass{ximera}
\outcome{Theorem environments.}
\typeout{Start loading xmPreamble.tex}%


\newcommand{\ximera}{Ximera}
\usepackage{lipsum}
% Add here extra macro's that are loaded automatically by all documents of claas 'ximera' or 'xourse' in this repo

%%
%%  Example:
%%
% \newcommand{\R}{\mathbb{R}
\author{Bart Snapp \and Rodney Austin}

\title{Theorem-like environments}


\begin{document}
\begin{abstract}
  Examples of the theorem environments.
\end{abstract}
\maketitle


But, \ximera\ provides a number of theorm-like environments.

\begin{theorem}
 \lipsum[1][1-3]
 \begin{proof}
  \lipsum[1][1-3]
\[
\frac{\partial V}{\partial t} + \frac{1}{2} \sigma^2 S^2 \frac{\partial^2 V}{\partial S^2} + r S \frac{\partial V}{\partial S} - r V = \answer{0}
\]
\lipsum[1][1-3]
 \end{proof}
\end{theorem}


\begin{theorem}[My theorem]
  \lipsum[1][1-3]
  \begin{proof}
    \lipsum[1][1-3]
    \paragraph{Case 1:}
    \lipsum[1][1-2]
    \[
\begin{pmatrix}
x & 2x & 3x \\
4x & 5x & 6x \\
7x & 8x & 9x
\end{pmatrix}
\begin{pmatrix}
a \\
b \\
c
\end{pmatrix}
=
\begin{pmatrix}
\answer{ax + 2b x+ 3cx} \\
\answer{4ax + 5bx + 6cx} \\
\answer{7a x+ 8bx + 9cx}
\end{pmatrix}
\]
\lipsum[1][1-1]
\paragraph{Case2:}
\lipsum[1][1-3]


  \end{proof}
\end{theorem}

\begin{algorithm}
  \lipsum[1][1-3]
\end{algorithm}

\begin{axiom}
  \lipsum[1][1-3]
\end{axiom}

\begin{claim}
  \lipsum[1][1-3]
\end{claim}

\begin{conclusion}
  \lipsum[1][1-3]
\end{conclusion}

\begin{condition}
  \lipsum[1][1-3]
\end{condition}

\begin{conjecture}
  \lipsum[1][1-3]
\end{conjecture}

\begin{corollary}
  \lipsum[1][1-3]
\end{corollary}

\begin{criterion}
  \lipsum[1][1-3]
\end{criterion}

\begin{definition}
  \lipsum[1][1-3]
\end{definition}

\begin{example}
  \lipsum[1][1-3]
  \begin{solution}
    \lipsum[1][1-3]
    \begin{align*}
      ax^2 + bx + c &= 0 \\
      x^2 + \frac{b}{a}x + \frac{c}{a} &= 0 \\
      x^2 + \frac{b}{a}x &= -\frac{c}{a} \\
      x^2 + \frac{b}{a}x + \left(\frac{b}{2a}\right)^2 &= \left(\frac{b}{2a}\right)^2 - \frac{c}{a} \\
      \left(x + \frac{b}{2a}\right)^2 &= \frac{b^2 - 4ac}{4a^2} \\
      x + \frac{b}{2a} &= \pm \frac{\sqrt{\answer{b^2 - 4ac}}}{2a} \\
      x &= \frac{-b \pm \sqrt{b^2 - 4ac}}{2a}
      \end{align*}
    \lipsum[1][1-3]
  \end{solution}
\end{example}

\begin{explanation}
  \lipsum[1][1-3]
\end{explanation}

\begin{fact}
  \lipsum[1][1-3]
\end{fact}

\begin{formula}
  \lipsum[1][1-3]
\end{formula}

\begin{idea}
  \lipsum[1][1-3]
\end{idea}

\begin{lemma}
  \lipsum[1][1-3]
  \begin{proof}
    \lipsum[1][1-3]
  \end{proof}
\end{lemma}

\begin{model}
  \lipsum[1][1-3]
\end{model}

\begin{notation}
  \lipsum[1][1-3]
\end{notation}

\begin{observation}
  \lipsum[1][1-3]
\end{observation}

\begin{paradox}
  \lipsum[1][1-3]
  \begin{align*}
    e^x &= 1 + x + \frac{x^2}{2!} + \frac{x^3}{3!} + \cdots \\
    &= 1 + \answer{x} + \frac{\answer{x^2}}{2} + \frac{\answer{x^3}}{6} + \cdots
    \end{align*}
    \lipsum[1][1-3]
\end{paradox}

\begin{procedure}
  \lipsum[1][1-3]
\end{procedure}

\begin{proposition}
  \lipsum[1][1-3]
  \begin{proof}
    \lipsum[1][1-3]
  \end{proof}
\end{proposition}

\begin{remark}
  \lipsum[1][1-3]
\end{remark}

\begin{summary}
  \lipsum[1][1-3]
\end{summary}

\begin{template}
  \lipsum[1][1-3]
\end{template}

\begin{warning}
  \lipsum[1][1-3]
\end{warning}


\end{document}
