\documentclass{ximera}
\title{Template}

%\typeout{Start loading xmPreamble.tex}%


\newcommand{\ximera}{Ximera}
\usepackage{lipsum}
% Add here extra macro's that are loaded automatically by all documents of claas 'ximera' or 'xourse' in this repo

%%
%%  Example:
%%
% \newcommand{\R}{\mathbb{R}% Only add a preamble file if it is actually necessary for the demo/test.
\begin{document}
\begin{abstract}
    A testbed file Template.
\end{abstract}
\maketitle

{{\Huge \bfseries Last Updated: \today}} \\

If you have come to this page, then the test activity you were looking into either hasn't had a testbed activity implemented for it yet, 
or it doesn't (as of the last update of this repo) have any commands, environments, or any other aspects that require testing.

\section{Basic Usage}

This is a template file. The Basic Usage section should be used to describe the \textbf{basic} usage 
of the commands/environments/etc (refered to hereafter as ``DemoCom'' for simplicity) being tested/demonstrated in the file. 

In particular, it should explain where/how the DemoCom should/must be placed (e.g. \verb|\answer| must be in mathemode 
and the abstract must be after begin document, not the preamble). It should also explain the most basic usecase, and any 
relevant optional parameters being tested/demo'ed in the file.

\section{Intended Outcome of Test}

The intended outcome section should show what the output of the demo case looks like, without actually using the DemoCom itself.
Any inconsistencies in formating/display should be noted as well as possible here as well, but it is important to not use the 
DemoCom itself so that we have an objective ``anticipated output'' to use as a comparison to what is \textit{actually} output in
test. 

\section{Start of Test/Demo Area}

This should contain the actual test/demonstration of DemoCom, which can then be compared to the description above to look for 
unexpected behavior.

\hrulefill


\end{document}