\documentclass{ximera}
\title{Validator Activity 2}
\date{Last Updated: 6/5/2025}

%
\usepackage{tabularray}

\ExplSyntaxOn

\__tblr_keys_define:nn { table/inner }
  {
    xmcolhead .code:n = \__tblr_keys_gput:nn { xmcolhead } {#1},
  }

\int_new:N \lTblrxmColHeadInt
\int_new:N \lTblrxmColDiffInt
\int_new:N \lTblrxmRowDiffInt
% \int_new:N \lTblrxmCellRowSpanInt
\int_new:N \lTblrxmColGroupCheck
\int_new:N \ColSpanEx
\tl_new:N \ColSpanEx_tl

\int_new:N \lTblrTestingInt
\prop_new:N \l__tblr_testing_prop

\cs_new:Npn \prop_get_to_int:NnN #1 #2 #3
  {
    \int_set:Nn #3 { \prop_item:Nn #1 { #2 } }
  }  

\cs_new_protected:Npn \__tblr_build_col_head_foot:
  {
    %% \lTblrColHeadInt can not be empty, so we append '+ 0'.
    \int_set:Nn \lTblrxmColHeadInt
      { \__tblr_prop_item:ne { inner } { xmcolhead } + 0 }
    \int_compare:nNnTF { \lTblrxmColHeadInt } > { 0 }
      {
        \__tblr_build_one_table:nnNN {1} { \lTblrxmColHeadInt }
          \c_true_bool \c_true_bool
      }
  }

%% #1: data name; #2: data index 1; #3: data index 2; #4: key
\cs_new:Npn \__tblr_data_test_item:nnnn #1 #2 #3 #4
  {
      \intarray_item:cn { g__tblr_#1_ \int_use:N \gTblrLevelInt _intarray }
        { \__tblr_data_key_to_int:nnnn {#1} {#2} {#3} {#4} }
      
  }

\__tblr_keys_define:nn { table/inner }
  {
    caption .code:n = \__tblr_keys_gput:nn { caption } {#1},
    summary .code:n = \__tblr_keys_gput:nn { summary } {#1}
  }


\tl_new:N \l__tblr_caption_tl
\str_new:N \l__tblr_caption_str

\tl_new:N \l__tblr_summary_tl
\str_new:N \l__tblr_summary_str

\cs_new:Npn \l__tblr_write_caption: #1
{
   \tl_set:Nn \l__tblr_caption_tl
      { \__tblr_prop_item:ne { inner } { caption }  }
   \str_set:Ne \l__tblr_caption_str
      { \tl_to_str:e \l__tblr_caption_tl }
    \tl_set:Nn \l__tblr_summary_tl
      { \__tblr_prop_item:ne { inner } { summary }  }
   \str_set:Ne \l__tblr_summary_str
      { \tl_to_str:e \l__tblr_summary_tl }
}

\int_new:N \l__tlbr_number_cols_int

\tl_new:N \l__tblr_colgroup_tl 

\ExplSyntaxOff

\begin{document}
\begin{abstract}
    Another testbed file for Validators
\end{abstract}
\maketitle

{{\Huge \bfseries Last Updated: \today}} \\

\section{Basic Usage}

This file demonstrates the following validators: factorCheck
This is implemented using the \verb|validator=factorCheck| as an optional parameter in the \verb|\answer| command.

\section{Intended Outcome of Test}
    \subsection*{factorCheck}
        This validator checks to see if the provided ``factored form'' from the student 
        is actually factored in a similar way to the author-provided ``factored form''.
        
        In the test/demo area are a number of problems that should demonstrate the factoring. The answers should be implied by the prompt, 
        but the answers (in order) are: $(x^2-4)(x-4)$, $(x-2)(x+2)(x-4)$, $(x-1)^3(x+1)(x^2-1)$

\section{Start of Test/Demo Area}

    \subsection*{factorCheck}
        \begin{problem}
            Consider the polynomial $x^3 - 4x^2 - 4x + 16$. You might want the student to just do the factor by grouping step 
            and want them to enter in $(x^2-4)(x-4)$. 
            Try trying in the full polynomial versus a fully factored version, 
            versus the desired version: $\answer[validator=factorCheck]{(x^2-4)(x-4)}$
        \end{problem}

        \begin{problem}
            Now, let's say you want them to fully factor, not just factor by grouping. 
            Try the factor by grouping version versus unfactored vs fully factored here: $\answer[validator=factorCheck]{(x-2)(x+2)(x-4)}$
        \end{problem}
        
        \begin{problem}
            Adding another problem here that deliberately has some repeated factors to show they work. 
            The answer should be $(x-1)^3(x+1)(x^2-1)$.
            
            $\answer[validator=factorCheck]{(x-1)^3(x+1)(x^2-1)}$
        \end{problem}


\hrulefill
\end{document}