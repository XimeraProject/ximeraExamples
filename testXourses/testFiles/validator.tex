\documentclass{ximera}
\title{Validator Activity 1}
\date{Last Updated: 6/5/2025}

%\typeout{Start loading xmPreamble.tex}%


\newcommand{\ximera}{Ximera}
\usepackage{lipsum}
% Add here extra macro's that are loaded automatically by all documents of claas 'ximera' or 'xourse' in this repo

%%
%%  Example:
%%
% \newcommand{\R}{\mathbb{R}
\begin{document}
\begin{abstract}
    A testbed file for Validators
\end{abstract}
\maketitle

{{\Huge \bfseries Last Updated: \today}} \\

\section{Basic Usage}

This file demonstrates the following validators: equalMaterix, multiAns, sameDerivative.
These are all implemented using the \verb|validator=(name of validator)| as an optional parameter in the \verb|\answer| command.

\section{Intended Outcome of Test}

    \subsection*{multiAns}
        This validator allows an instructor to provide a list of acceptable answers for a given answer prompt.

        The testing area for multiAns should show a problem with the statement: 
        " The following answer prompt will accept any of the following functions as correct: $\sin(t)$, $\tan(x)$, $e^\pi$, $21x^3 + 1$, $ex + \pi$. "
        and then (as implied) an \verb|\answer| box that should accept any of the listed functions/values as correct, and reject any others.

    \subsection*{sameDerivative}
        The testing area for multiAns should show a problem with the statement: 
        "This problem shows the 'same derivative' validator. It's intended to be used to test the result of an indefinite integral, so it requires the ``$+C$'' at the end, and (should) mark any answer wrong that doesn't have it. Note that the $C$ is case-sensitive. \\
            
            Enter in any answer whose derivative is $x^2 + \sin(x) - 3$ (and don't forget the $+C$, but notice you can also add random constants to it too)."
        and then (as implied) it will have the equation $\int x^2 + \sin(x) - 3 dx =$ followed by an answer box that will take any (correct) answer, i.e. the 
        indefinite integral with a +C and any other constant added.

    \subsection*{equalMatrix}

        This validator is designed to ensure that a matrix submitted by a student is the same as the matrix submitted by the instructor. This also establishes the format for how matricies should be entered into the answer box and answer command.
    
        NOTE:: This validator is still under construction. Currently this validator (seems to) work on algebraic entries of any kinda that Xronos normally supports.
        
        Specifically, a matrix should be entered into the answer box in the following format:
        
        $[ [r1c1,r1c2,r1c3],[r2c1,r2c2,r2c3],[r3c1,r3c2,r3c3] ]$.
        
        So, if a student wishes to enter the matrix:
        
        $\left[\begin{matrix}
        1 & 2 & 3 \\
        4 & 5 & 6 \\
        7 & 8 & 9
        \end{matrix}\right]$
        
        They would enter into the answer prompt: $[ [1,2,3], [4,5,6], [7,8,9] ]$.\\

        In the test/demo area, you should see a problem with the prompt:
            "The following answer box should accept the matrix:
            $\left[\begin{matrix}
            a^2 & 2 & 3 \\
            4 & x^2 & 6 \\
            7 & 8 & y-x+e^x
            \end{matrix}\right]$"

            And then an \verb|\answer| box that should mark correct the answer $[ [a^2,2,3], [4,x^2,6], [7,8,y-x+e^x] ]$ and no other possibilities.


\section{Start of Test/Demo Area}

    \subsection*{multiAns}
        \begin{problem}
            The following answer prompt will accept any of the following functions as correct: $\sin(t)$, $\tan(x)$, $e^\pi$, $21x^3 + 1$, $ex + \pi$.
            \[
                \answer[validator=multiAns]{[\sin(t),\tan(x),e^\pi,21x^3 + 1,ex + \pi]}
            \]
        \end{problem}

    \subsection*{sameDerivative}

        \begin{problem}
            This problem shows the 'same derivative' validator. It's intended to be used to test the result of an indefinite integral, so it requires the ``$+C$'' at the end, and (should) mark any answer wrong that doesn't have it. Note that the $C$ is case-sensitive. \\
            
            Enter in any answer whose derivative is $x^2 + \sin(x) - 3$ (and don't forget the $+C$, but notice you can also add random constants to it too).
            \[
                \int x^2 + \sin(x) - 3 dx = \answer[validator=sameDerivative]{\frac{1}{3}x^3 - \cos(x) - 3x + C}
            \]
            
            Sidenote: You can avoid the problem of the $+C$ if you don't put it in the expected answer box, which will then (correctly) mark an answer wrong that \textit{does} include a $+C$. So in a sense you can force the student to include it, or not, with this validator. Although only checking the derivatives to match has obvious issues for non indefinite integrals.
        \end{problem}

    \subsection*{equalMatrix}
        \begin{problem}
            The following answer box should accept the matrix:
            
            $\left[\begin{matrix}
            a^2 & 2 & 3 \\
            4 & x^2 & 6 \\
            7 & 8 & y-x+e^x
            \end{matrix}\right]$
            
            $\answer[validator=equalMatrix]{[ [a^2,2,3], [4,x^2,6], [7,8,y-x+e^x] ]}$
        \end{problem}


\hrulefill
\end{document}