\documentclass{article}

\usepackage{hyperref}  % optional, better caption formatting
\usepackage{amsmath}
\usepackage{lipsum}

\title{Labels and refs (in documentclass article)}

\begin{document}

\maketitle

\section{Introduction}
\label{sec:intro}

Welcome to the demo. In \autoref{sec:math}, we will discuss some math.

\section{Mathematics Example}
\label{sec:math}

Consider the Pythagorean theorem:

\begin{equation}
    a^2 + b^2 = c^2 \label{p1}
\end{equation}
\begin{equation}
    a^2 + b^2 = c^2 \label{eq:p2}
\end{equation}


We can refer to the first one using \verb|\ref|: Equation~\ref{p1}, or with \verb|\eqref|: \eqref{p1}.

We can refer to the first one using \verb|\ref|: Equation~$\ref{p1}$, or with \verb|\eqref|: $\eqref{p1}$.

We can refer to the first one using \verb|\ref|: Equation~\ref{eq:p2}, or with \verb|\eqref|: \eqref{eq:p2}.

We can refer to the first one using \verb|\ref|: Equation~$\ref{eq:p2}$, or with \verb|\eqref|: $\eqref{eq:p2}$.

\lipsum[1]

\lipsum[12]

\lipsum[10]

\lipsum[10]
\lipsum[10]
\lipsum[10]
\lipsum[10]


\end{document}
